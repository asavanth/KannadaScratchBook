\chapter{ಮರುಕಳಿಸು (\textenglish{Loops})}
\SampleProgram

\section{ಇಷ್ಟು ಬಾರಿ ಮರುಕಳಿಸು }

\begin{center}
\begin{figure}
\begin{multicols}{2}
\begin{Scratch}[1]
\scbox{\cb[w]{10}  ಹೆಜ್ಜೆ ಮುಂದೆ ಹೋಗು}{motion}
\scbox{\cb[w]{20}  ಹೆಜ್ಜೆ ಮುಂದೆ ಹೋಗು}{motion}
\end{Scratch}


\begin{tikzpicture}
\node[doc] (x) (inst1){
10  ಹೆಜ್ಜೆ ಮುಂದೆ ಹೋಗು\\
20  ಹೆಜ್ಜೆ ಮುಂದೆ ಹೋಗು
};
\end{tikzpicture}
\end{multicols}

\caption{}
\label{writing}
\end{figure}
\end{center}


\begin{center}
\begin{figure}
\begin{multicols}{2}
\begin{Scratch}[1]
\boucle{\cb[w]{4} ಮರುಕಳಿಸು}{1}{1}
\scbox{\cb[w]{20}  ಹೆಜ್ಜೆ ಮುಂದೆ ಹೋಗು}{motion}
\scbox{\cb[w]{50}  ಹೆಜ್ಜೆ ಮುಂದೆ ಹೋಗು}{motion}
\end{Scratch}


\begin{tikzpicture}
\node[doc] (x) (inst1){
4 ಮರುಕಳಿಸು\\
\hspace{0.6cm}20  ಹೆಜ್ಜೆ ಮುಂದೆ ಹೋಗು\\
50  ಹೆಜ್ಜೆ ಮುಂದೆ ಹೋಗು
};
\end{tikzpicture}
\end{multicols}

\caption{}
\label{writing}
\end{figure}
\end{center}
\section{ಯಾವಾಗಲು ಮರುಕಳಿಸು }
\section{ಅಭ್ಯಾಸ }