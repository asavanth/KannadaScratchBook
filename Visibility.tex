\chapter{ಧ್ವನಿ-ದೃಶ್ಯ}

\section{ಧ್ವನಿ }

\begin{figure}[h]
\begin{center}
\begin{multicols}{3}
\begin{Scratch}[1]
\scbox{\rb{ಮಿಯವ್} ಶಬ್ಧ ಕೇಳಿಸು}{sound}
\end{Scratch}
\begin{Scratch}[1]
\scbox{\cb[w]{60 \scriptsize$\blacktriangledown$} ಲಯದಲ್ಲಿ  \cb[w]{1}   ಸಂಗೀತದ ಸ್ವರ ಕೇಳಿಸು}{sound}
\end{Scratch}
\begin{Scratch}[1]
\scbox{\cb[w]{1 \scriptsize$\blacktriangledown$} ಗೆ  ವಾದ್ಯವನ್ನು ಹೊಂದಿಸು}{sound}
\end{Scratch}

\end{multicols}
\end{center}
\caption{ಸ್ಪ್ರಯ್ಟ್ನಿಂದ ಹೇಳಿಸುವುದು}
\label{vis_sound}
\end{figure}

\section{ಹೆಲೊ ಎಂದು ಹೇಳು }
\begin{figure}[h]
\begin{center}
\begin{multicols}{2}
\begin{Scratch}[1]
\beginbox{}
\scbox{\rb[w]{ಹೆಲೋ} ಎಂದು ಹೇಳು}{looks}
\end{Scratch}

\begin{Scratch}[1]
\beginbox{}
\scbox{\rb[w]{ಹೆಲೋ} ಎಂದು \cb[w]{2}  ಸೆಕೆಂಡಿನಲ್ಲಿ ಹೇಳು}{looks}
\end{Scratch}

\end{multicols}

\begin{tikzpicture}
\node at (0,0) [rectangle callout,rounded corners, draw=gray!60,  line width=3pt,  anchor=south west, callout relative pointer={(-0.5cm, -0.5cm)}](tell){\Large{ಹೆಲೋ}}; 
\node at ([shift={(0.8cm,0.2cm)}]tell.pointer)[anchor=north east, rotate=0, opacity=1](s1){\Scratchy[0.2]};
\end{tikzpicture}
\end{center}
\caption{ಸ್ಪ್ರಯ್ಟ್ನಿಂದ ಹೇಳಿಸುವುದು}
\label{vis_hello}
\end{figure}

\section{ತೋರಿಸು, ಬಚ್ಚಿಡು}
\begin{figure}[h]
\begin{center}
\begin{multicols}{2}
\begin{Scratch}[1]
\scbox{ಬಚ್ಚಿಡು}{looks}
\end{Scratch}
\begin{Scratch}[1]
\scbox{ತೋರಿಸು}{looks}
\end{Scratch}

\end{multicols}
\end{center}
\caption{ಸ್ಪ್ರಯ್ಟ್ಅನ್ನು ಕಣ್ಮರೆ ಮಾಡುವುದು}
\label{vis_hide}
\end{figure}


\section{ಉಡುಪು ಬದಲಾಯಿಸು}

\begin{figure}[h]
\begin{center}
\begin{Scratch}[1]
\scbox{\rb{ಉಡುಪು1} ಗೆ ಉಡುಪನ್ನು ಬದಲಾಯಿಸು}{sound}
\end{Scratch}
\end{center}
\caption{ಸ್ಪ್ರಯ್ಟ್ ಉಡುಪನ್ನು ಬದಲಾಯಿಸುವುದು}
\label{vis_looks}
\end{figure}

\section{ಅಭ್ಯಾಸ }

