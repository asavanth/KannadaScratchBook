\newpage
\begin{center}
\Large{ಕಂಪ್ಯೂಟರ್ ಸೆಟಪ್ ಮಾಡಿಕೊಳ್ಳುವುದು}
\end{center}
\vspace{1cm}
ಸ್ಕ್ರಾಚ್ ಪ್ರಾರಂಭಿಸಿದಾಗ ಹೀಗೆ ಕಾಣಿಸುತ್ತದೆ. \par

ಪ್ರೋಗ್ರಾಂ ತಯಾರಿಯ ಸ್ಥಳದಲ್ಲಿ ಆಜ್ಞೆಗಳನ್ನು ಜೋಡಿಸಿ ಪ್ರೋಗ್ರಾಂ ತಯಾರಿಸಬೇಕು. ಪ್ರೋಗ್ರಾಂನಲ್ಲಿ ನೀಡಿದ ಆಜ್ಞೆಯ ಪ್ರಕಾರ ಸ್ಪ್ರೈಟ್ ಕೆಲಸ ಮಾಡುತ್ತದೆ. ಉದಾಹರಣೆಗೆ “ಹತ್ತು ಹೆಜ್ಜೆ ಮುಂದೆ ಹೋಗು” ಎಂದು ಆಜ್ಞೆ ಮಾಡಿದರೆ ಅದು ಹತ್ತು ಹೆಜ್ಜೆ ಮುಂದೆ ಹೋಗುತ್ತದೆ. ನಂತರ “90 ಡಿಗ್ರಿ ಬಲಕ್ಕೆ ತಿರುಗು” ಎಂದರೆ ಬಲಕ್ಕೆ ತಿರುಗುತ್ತದೆ. ಇನ್ನೂ ಹಲವಾರು ವಿಧದ ಆಜ್ಞೆಗಳಿವೆ. ಅವುಗಳನ್ನೆಲ್ಲ ಮುಂದಕ್ಕೆ ಒಂದೊಂದಾಗಿ ತಿಳಿಯೋಣ.
\vspace{1cm}

