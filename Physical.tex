
\chapter{ಭೌತಿಕ ಗಣಕ}

\section{ಎಲ್.ಇ.ಡಿ ಉರಿಸುವುದು}
\begin{figure}[h]
\LED{red}\LED{green}\LED{yellow}\LED{gray}
\caption{ಉದಾಹರಣೆ ಪ್ರೋಗ್ರಾಂ ಎಲ್.ಇ.ಡಿ}
\end{figure}

\begin{figure}[h]
\begin{Scratch}[1]
\beginbox{}
\scbox{ಲೇಖನಿಯುಕ್ತ}{pen}
\boucle{\cb[w]{10} ಮರುಕಳಿಸು}{4}{1}
\scbox{\cb[w]{10}  ಹೆಜ್ಜೆ ಮುಂದೆ ಹೋಗು}{motion}
\scbox{\cb[w]{1}  ಸೆಕೆಂಡುಗಳಷ್ಟು   ಕಾಯಬೇಕು}{control}
\scbox{\cb[w]{10} ಡಿಗ್ರಿಯಷ್ಟು ತಿರುಗು}{motion}
\turnbox{g}{180}{ಡಿಗ್ರಿಯಷ್ಟು ತಿರುಗು}
\boucle{\cb[w]{10} ಮರುಕಳಿಸು}{6}{1}
\scbox{\cb[w]{10}  ಹೆಜ್ಜೆ ಮುಂದೆ ಹೋಗು}{motion}
\scbox{\cb[w]{1}  ಸೆಕೆಂಡುಗಳಷ್ಟು   ಕಾಯಬೇಕು}{control}
\scbox{\cb[w]{10} ಡಿಗ್ರಿಯಷ್ಟು ತಿರುಗು}{motion}
\turnbox{g}{180}{ಡಿಗ್ರಿಯಷ್ಟು ತಿರುಗು}
\turnbox{1}{180}{ಡಿಗ್ರಿಯಷ್ಟು ತಿರುಗು}
\scbox{ಲೇಖನಿಮುಕ್ತ}{pen}
\end{Scratch}
\caption{ಉದಾಹರಣೆ ಪ್ರೋಗ್ರಾಂ 1}
\end{figure}
\section{ಗ್ರಹಿಸಿ ಪ್ರತಿಕ್ರಿಯಿಸುವುದು} 
\SampleProgram

\section{ಅಭ್ಯಾಸ }
