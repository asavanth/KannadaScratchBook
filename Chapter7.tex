\chapter{ಆಪರೇಟರ್ಗಳು}
\begin{figure}[h]
\begin{Scratch}[1]
\beginbox{}
\scbox{\cb[w]{10} ಹತ್ತು ಹೆಜ್ಜೆ ಮುಂದೆ ಹೋಗು}{motion}
\scbox{\cb[w]{1}  ಸೆಕೆಂಡುಗಳಷ್ಟು   ಕಾಯಬೇಕು}{control}
\scbox{\cb[w]{10} ಡಿಗ್ರಿಯಷ್ಟು ತಿರುಗು}{motion}
\end{Scratch}
\caption{ಉದಾಹರಣೆ ಪ್ರೋಗ್ರಾಂ 6}
\end{figure}

\section{ಅಂಕಗಣಿತದ ಆಯೋಜಕರು}
ಇದನ್ನು ಪರಿಶೀಲಿಸಲು ಮೇಲಿನ ಚಿತ್ರದಲ್ಲಿ ತೊರಿಸಿರುವಂತೆ ಆಜ್ಞೆಗಳನ್ನು ಎಳೆದು ತಂದು ಪ್ರೋಗ್ರಾಂ ತಯಾರಿಯ ಸ್ಥಳದಲ್ಲಿ ಇಡೋಣ. ಇದನ್ನು ಚಾಲತಿ ಮಾಡಿ ನೋಡಿ. “ಒಂದುವೇಳೆ” ಆಜ್ಞೆ ಸರಿ ಬರುವಂತೆ ಹಾಗು ತಪ್ಪಾಗುವಂತೆ ಮಾಡಿ ಪ್ರೋಗ್ರಾಂ ಚಾಲತಿಯಲ್ಲಿನ ವೆತ್ಯಾಸವನ್ನು ಮನವರಿಕೆ ಮಾಡಿಕೊಳ್ಳಿ. ಇಲ್ಲಿ, ಕೆಳಕಂಡ ಚಿತ್ರದಂತೆ, ಸ್ಪ್ರೈಟ್ ಕೆಲವು ಹೆಜ್ಜೆ ಮುಂದೆ ಹೋಗುವಂತೆ ಮಾಡಿದರೆ ಸ್ಪ್ರೈಟ್ ವರ್ತುಲಾಕಾರದಲ್ಲಿ ತಿರುಗುವುದನ್ನು ನೋಡಬಹುದು

\section{ಹೋಲಿಕೆ ಆಯೋಜಕರು}


\section{ತಾರ್ಕಿಕ ಆಯೋಜಕರು}

\section{ಅಭ್ಯಾಸ }