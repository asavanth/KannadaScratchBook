\documentclass[16pt]{book}
\usepackage[a4paper]{geometry}
\usepackage{fontspec}
\usepackage{xcolor}
\newfontfamily\kannadafont{Noto Sans Kannada}
\usepackage{polyglossia}
\setmainlanguage{kannada}
\setotherlanguages{english}

\usepackage{xcolor}
\usepackage{array}
\usepackage{multirow}
\usepackage{lastpage}
\usepackage{verbatim}
%\usepackage{scratch3}
\usepackage[nomessages]{fp}
\usepackage{calc}
\usepackage{xstring}
\usepackage[alpine]{ifsym}%pour avoir VarFlag comme
\usepackage{ifthen}
\usepackage{multido}
\usepackage{xargs}
\usepackage{tikz}
\usepackage{pgf}
\usetikzlibrary{calc, fadings}
\usepackage{amsfonts,amsmath,amssymb,mathrsfs,amsthm}
\usepackage{lmodern}
\usepackage{ScratchX}


\begin{document}
\begin{titlepage} % Suppresses headers and footers on the title page
	
	\centering % Centre everything on the title page
	
	%------------------------------------------------
	%	Top rules
	%------------------------------------------------
	
	\rule{\textwidth}{1pt} % Thick horizontal rule
	
	\vspace{2pt}\vspace{-\baselineskip} % Whitespace between rules
	
	\rule{\textwidth}{0.4pt} % Thin horizontal rule
	
	\vspace{0.1\textheight} % Whitespace between the top rules and title
	
	%------------------------------------------------
	%	Title
	%------------------------------------------------
	
	\textcolor{red}{ % Red font color
		{\Huge\textbf{\color{blue}{ಕನ್ನಡದಲ್ಲಿ ಸ್ಕ್ರಾಚ್ ಪ್ರೋಗ್ರಾಮ್ಮಿಂಗ್}}
}
}
	
	\vspace{0.025\textheight} % Whitespace between the title and short horizontal rule
	
	\rule{0.3\textwidth}{0.4pt} % Short horizontal rule under the title
	
	\vspace{0.1\textheight} % Whitespace between the thin horizontal rule and the author name
	
	%------------------------------------------------
	%	Author
	%------------------------------------------------
	\begin{center}
	{\Large \textsc{ಡಾ|| ಯು. ಬಿ. ಪವನಜ}}\\
	ಮತ್ತು\\
	{\Large \textsc{ಡಾ|| ಆನಂದ್ ಸಾವಂತ್}} % Author name
	\end{center}
	
	\vspace{3cm} % Whitespace between the author name and publisher
	\includegraphics[scale=0.4]{KKali_logo.pdf}
	\includegraphics[scale=0.4]{KKali_logo.pdf}
	\includegraphics[scale=0.4]{KKali_logo.pdf}
	\vfill
	%------------------------------------------------
	%	Publisher
	%------------------------------------------------
	
	%{\large\textcolor{red}{\plogo}}\\[0.5\baselineskip] % Publisher logo
	
	{\large\textsc{ಪಬ್ಲಿಶರ್}} % Publisher
	
	\vspace{0.1\textheight} % Whitespace under the publisher text
	
	%------------------------------------------------
	%	Bottom rules
	%------------------------------------------------
	
	\rule{\textwidth}{0.4pt} % Thin horizontal rule
\begin{small}
\color{gray}{(ಸೂಚನೆ: ಈ ಟ್ಯುಟೋರಿಯಲ್‌ ಪುಸ್ತಕಕ್ಕೆ ಬಳಸಿದ್ದು ಸ್ಕ್ರಾಚ್ 1.4 ಆವೃತ್ತಿ)}\par
\end{small}

	\vspace{8pt}\vspace{-\baselineskip} % Whitespace between rules
	
	\rule{\textwidth}{1pt} % Thick horizontal rule
	
\end{titlepage}

\begin{center}
\Large\textbf{\color{blue}{ಮುನ್ನುಡಿ}}
\end{center}
\vspace{1cm}
ಮುನ್ನುಡಿ ಕೃತಿಯ ಲೇಖಕನಿಂದ ಬರೆಯಲಾದ ಒಂದು ಪುಸ್ತಕ ಅಥವಾ ಇತರ ಸಾಹಿತ್ಯ ಕೃತಿಗೆ ಪೀಠಿಕೆ. ಒಬ್ಬ ಬೇರೆ ವ್ಯಕ್ತಿಯು ಬರೆದ ಪರಿಚಯಾತ್ಮಕ ಪ್ರಬಂಧಕ್ಕೆ ಪ್ರಸ್ತಾವನೆ ಎಂದು ಕರೆಯಲಾಗುತ್ತದೆ ಮತ್ತು ಇದು ಲೇಖಕನ ಮುನ್ನುಡಿಗಿಂತ ಮೊದಲು ಬರುತ್ತದೆ. ಮುನ್ನುಡಿಯು ಹಲವುವೇಳೆ ಆ ಸಾಹಿತ್ಯ ಕೃತಿಯಲ್ಲಿ ನೆರವಾದವರಿಗೆ ವಂದನೆಗಳೊಂದಿಗೆ ಅಂತ್ಯಗೊಳ್ಳುತ್ತದೆ.ಮುನ್ನುಡಿ ಕೃತಿಯ ಲೇಖಕನಿಂದ ಬರೆಯಲಾದ ಒಂದು ಪುಸ್ತಕ ಅಥವಾ ಇತರ ಸಾಹಿತ್ಯ ಕೃತಿಗೆ ಪೀಠಿಕೆ. ಒಬ್ಬ ಬೇರೆ ವ್ಯಕ್ತಿಯು ಬರೆದ ಪರಿಚಯಾತ್ಮಕ ಪ್ರಬಂಧಕ್ಕೆ ಪ್ರಸ್ತಾವನೆ ಎಂದು ಕರೆಯಲಾಗುತ್ತದೆ ಮತ್ತು ಇದು ಲೇಖಕನ ಮುನ್ನುಡಿಗಿಂತ ಮೊದಲು ಬರುತ್ತದೆ. ಮುನ್ನುಡಿಯು ಹಲವುವೇಳೆ ಆ ಸಾಹಿತ್ಯ ಕೃತಿಯಲ್ಲಿ ನೆರವಾದವರಿಗೆ ವಂದನೆಗಳೊಂದಿಗೆ ಅಂತ್ಯಗೊಳ್ಳುತ್ತದೆ.
\vspace{1cm}

ಮುನ್ನುಡಿ ಕೃತಿಯ ಲೇಖಕನಿಂದ ಬರೆಯಲಾದ ಒಂದು ಪುಸ್ತಕ ಅಥವಾ ಇತರ ಸಾಹಿತ್ಯ ಕೃತಿಗೆ ಪೀಠಿಕೆ. ಒಬ್ಬ ಬೇರೆ ವ್ಯಕ್ತಿಯು ಬರೆದ ಪರಿಚಯಾತ್ಮಕ ಪ್ರಬಂಧಕ್ಕೆ ಪ್ರಸ್ತಾವನೆ ಎಂದು ಕರೆಯಲಾಗುತ್ತದೆ ಮತ್ತು ಇದು ಲೇಖಕನ ಮುನ್ನುಡಿಗಿಂತ ಮೊದಲು ಬರುತ್ತದೆ. ಮುನ್ನುಡಿಯು ಹಲವುವೇಳೆ ಆ ಸಾಹಿತ್ಯ ಕೃತಿಯಲ್ಲಿ ನೆರವಾದವರಿಗೆ ವಂದನೆಗಳೊಂದಿಗೆ ಅಂತ್ಯಗೊಳ್ಳುತ್ತದೆ.ಮುನ್ನುಡಿ ಕೃತಿಯ ಲೇಖಕನಿಂದ ಬರೆಯಲಾದ ಒಂದು ಪುಸ್ತಕ ಅಥವಾ ಇತರ ಸಾಹಿತ್ಯ ಕೃತಿಗೆ ಪೀಠಿಕೆ. ಒಬ್ಬ ಬೇರೆ ವ್ಯಕ್ತಿಯು ಬರೆದ ಪರಿಚಯಾತ್ಮಕ ಪ್ರಬಂಧಕ್ಕೆ ಪ್ರಸ್ತಾವನೆ ಎಂದು ಕರೆಯಲಾಗುತ್ತದೆ ಮತ್ತು ಇದು ಲೇಖಕನ ಮುನ್ನುಡಿಗಿಂತ ಮೊದಲು ಬರುತ್ತದೆ. ಮುನ್ನುಡಿಯು ಹಲವುವೇಳೆ ಆ ಸಾಹಿತ್ಯ ಕೃತಿಯಲ್ಲಿ ನೆರವಾದವರಿಗೆ ವಂದನೆಗಳೊಂದಿಗೆ ಅಂತ್ಯಗೊಳ್ಳುತ್ತದೆ.ಮುನ್ನುಡಿ ಕೃತಿಯ ಲೇಖಕನಿಂದ ಬರೆಯಲಾದ ಒಂದು ಪುಸ್ತಕ ಅಥವಾ ಇತರ ಸಾಹಿತ್ಯ ಕೃತಿಗೆ ಪೀಠಿಕೆ. ಒಬ್ಬ ಬೇರೆ ವ್ಯಕ್ತಿಯು ಬರೆದ ಪರಿಚಯಾತ್ಮಕ ಪ್ರಬಂಧಕ್ಕೆ ಪ್ರಸ್ತಾವನೆ ಎಂದು ಕರೆಯಲಾಗುತ್ತದೆ ಮತ್ತು ಇದು ಲೇಖಕನ ಮುನ್ನುಡಿಗಿಂತ ಮೊದಲು ಬರುತ್ತದೆ. ಮುನ್ನುಡಿಯು ಹಲವುವೇಳೆ ಆ ಸಾಹಿತ್ಯ ಕೃತಿಯಲ್ಲಿ ನೆರವಾದವರಿಗೆ ವಂದನೆಗಳೊಂದಿಗೆ ಅಂತ್ಯಗೊಳ್ಳುತ್ತದೆ.

\vspace{1cm}
ಮುನ್ನುಡಿ ಕೃತಿಯ ಲೇಖಕನಿಂದ ಬರೆಯಲಾದ ಒಂದು ಪುಸ್ತಕ ಅಥವಾ ಇತರ ಸಾಹಿತ್ಯ ಕೃತಿಗೆ ಪೀಠಿಕೆ. ಒಬ್ಬ ಬೇರೆ ವ್ಯಕ್ತಿಯು ಬರೆದ ಪರಿಚಯಾತ್ಮಕ ಪ್ರಬಂಧಕ್ಕೆ ಪ್ರಸ್ತಾವನೆ ಎಂದು ಕರೆಯಲಾಗುತ್ತದೆ ಮತ್ತು ಇದು ಲೇಖಕನ ಮುನ್ನುಡಿಗಿಂತ ಮೊದಲು ಬರುತ್ತದೆ. ಮುನ್ನುಡಿಯು ಹಲವುವೇಳೆ ಆ ಸಾಹಿತ್ಯ ಕೃತಿಯಲ್ಲಿ ನೆರವಾದವರಿಗೆ ವಂದನೆಗಳೊಂದಿಗೆ ಅಂತ್ಯಗೊಳ್ಳುತ್ತದೆ.
\vspace{1cm}
\newpage
\begin{center}
\Large\textbf{\color{blue}{ಸ್ವೀಕೃತಿ}}
\end{center}
\vspace{1cm}
ಮುನ್ನುಡಿ ಕೃತಿಯ ಲೇಖಕನಿಂದ ಬರೆಯಲಾದ ಒಂದು ಪುಸ್ತಕ ಅಥವಾ ಇತರ ಸಾಹಿತ್ಯ ಕೃತಿಗೆ ಪೀಠಿಕೆ. ಒಬ್ಬ ಬೇರೆ ವ್ಯಕ್ತಿಯು ಬರೆದ ಪರಿಚಯಾತ್ಮಕ ಪ್ರಬಂಧಕ್ಕೆ ಪ್ರಸ್ತಾವನೆ ಎಂದು ಕರೆಯಲಾಗುತ್ತದೆ ಮತ್ತು ಇದು ಲೇಖಕನ ಮುನ್ನುಡಿಗಿಂತ ಮೊದಲು ಬರುತ್ತದೆ. ಮುನ್ನುಡಿಯು ಹಲವುವೇಳೆ ಆ ಸಾಹಿತ್ಯ ಕೃತಿಯಲ್ಲಿ ನೆರವಾದವರಿಗೆ ವಂದನೆಗಳೊಂದಿಗೆ ಅಂತ್ಯಗೊಳ್ಳುತ್ತದೆ.ಮುನ್ನುಡಿ ಕೃತಿಯ ಲೇಖಕನಿಂದ ಬರೆಯಲಾದ ಒಂದು ಪುಸ್ತಕ ಅಥವಾ ಇತರ ಸಾಹಿತ್ಯ ಕೃತಿಗೆ ಪೀಠಿಕೆ. ಒಬ್ಬ ಬೇರೆ ವ್ಯಕ್ತಿಯು ಬರೆದ ಪರಿಚಯಾತ್ಮಕ ಪ್ರಬಂಧಕ್ಕೆ ಪ್ರಸ್ತಾವನೆ ಎಂದು ಕರೆಯಲಾಗುತ್ತದೆ ಮತ್ತು ಇದು ಲೇಖಕನ ಮುನ್ನುಡಿಗಿಂತ ಮೊದಲು ಬರುತ್ತದೆ. ಮುನ್ನುಡಿಯು ಹಲವುವೇಳೆ ಆ ಸಾಹಿತ್ಯ ಕೃತಿಯಲ್ಲಿ ನೆರವಾದವರಿಗೆ ವಂದನೆಗಳೊಂದಿಗೆ ಅಂತ್ಯಗೊಳ್ಳುತ್ತದೆ.
\vspace{1cm}

ಮುನ್ನುಡಿ ಕೃತಿಯ ಲೇಖಕನಿಂದ ಬರೆಯಲಾದ ಒಂದು ಪುಸ್ತಕ ಅಥವಾ ಇತರ ಸಾಹಿತ್ಯ ಕೃತಿಗೆ ಪೀಠಿಕೆ. ಒಬ್ಬ ಬೇರೆ ವ್ಯಕ್ತಿಯು ಬರೆದ ಪರಿಚಯಾತ್ಮಕ ಪ್ರಬಂಧಕ್ಕೆ ಪ್ರಸ್ತಾವನೆ ಎಂದು ಕರೆಯಲಾಗುತ್ತದೆ ಮತ್ತು ಇದು ಲೇಖಕನ ಮುನ್ನುಡಿಗಿಂತ ಮೊದಲು ಬರುತ್ತದೆ. ಮುನ್ನುಡಿಯು ಹಲವುವೇಳೆ ಆ ಸಾಹಿತ್ಯ ಕೃತಿಯಲ್ಲಿ ನೆರವಾದವರಿಗೆ ವಂದನೆಗಳೊಂದಿಗೆ ಅಂತ್ಯಗೊಳ್ಳುತ್ತದೆ.ಮುನ್ನುಡಿ ಕೃತಿಯ ಲೇಖಕನಿಂದ ಬರೆಯಲಾದ ಒಂದು ಪುಸ್ತಕ ಅಥವಾ ಇತರ ಸಾಹಿತ್ಯ ಕೃತಿಗೆ ಪೀಠಿಕೆ. ಒಬ್ಬ ಬೇರೆ ವ್ಯಕ್ತಿಯು ಬರೆದ ಪರಿಚಯಾತ್ಮಕ ಪ್ರಬಂಧಕ್ಕೆ ಪ್ರಸ್ತಾವನೆ ಎಂದು ಕರೆಯಲಾಗುತ್ತದೆ ಮತ್ತು ಇದು ಲೇಖಕನ ಮುನ್ನುಡಿಗಿಂತ ಮೊದಲು ಬರುತ್ತದೆ. ಮುನ್ನುಡಿಯು ಹಲವುವೇಳೆ ಆ ಸಾಹಿತ್ಯ ಕೃತಿಯಲ್ಲಿ ನೆರವಾದವರಿಗೆ ವಂದನೆಗಳೊಂದಿಗೆ ಅಂತ್ಯಗೊಳ್ಳುತ್ತದೆ.ಮುನ್ನುಡಿ ಕೃತಿಯ ಲೇಖಕನಿಂದ ಬರೆಯಲಾದ ಒಂದು ಪುಸ್ತಕ ಅಥವಾ ಇತರ ಸಾಹಿತ್ಯ ಕೃತಿಗೆ ಪೀಠಿಕೆ. ಒಬ್ಬ ಬೇರೆ ವ್ಯಕ್ತಿಯು ಬರೆದ ಪರಿಚಯಾತ್ಮಕ ಪ್ರಬಂಧಕ್ಕೆ ಪ್ರಸ್ತಾವನೆ ಎಂದು ಕರೆಯಲಾಗುತ್ತದೆ ಮತ್ತು ಇದು ಲೇಖಕನ ಮುನ್ನುಡಿಗಿಂತ ಮೊದಲು ಬರುತ್ತದೆ. ಮುನ್ನುಡಿಯು ಹಲವುವೇಳೆ ಆ ಸಾಹಿತ್ಯ ಕೃತಿಯಲ್ಲಿ ನೆರವಾದವರಿಗೆ ವಂದನೆಗಳೊಂದಿಗೆ ಅಂತ್ಯಗೊಳ್ಳುತ್ತದೆ.

\vspace{1cm}
ಮುನ್ನುಡಿ ಕೃತಿಯ ಲೇಖಕನಿಂದ ಬರೆಯಲಾದ ಒಂದು ಪುಸ್ತಕ ಅಥವಾ ಇತರ ಸಾಹಿತ್ಯ ಕೃತಿಗೆ ಪೀಠಿಕೆ. ಒಬ್ಬ ಬೇರೆ ವ್ಯಕ್ತಿಯು ಬರೆದ ಪರಿಚಯಾತ್ಮಕ ಪ್ರಬಂಧಕ್ಕೆ ಪ್ರಸ್ತಾವನೆ ಎಂದು ಕರೆಯಲಾಗುತ್ತದೆ ಮತ್ತು ಇದು ಲೇಖಕನ ಮುನ್ನುಡಿಗಿಂತ ಮೊದಲು ಬರುತ್ತದೆ. ಮುನ್ನುಡಿಯು ಹಲವುವೇಳೆ ಆ ಸಾಹಿತ್ಯ ಕೃತಿಯಲ್ಲಿ ನೆರವಾದವರಿಗೆ ವಂದನೆಗಳೊಂದಿಗೆ ಅಂತ್ಯಗೊಳ್ಳುತ್ತದೆ.
\tableofcontents

\newpage
\listoffigures

\newpage
\begin{center}
\Large{ಕಂಪ್ಯೂಟರ್ ಸೆಟಪ್ ಮಾಡಿಕೊಳ್ಳುವುದು}
\end{center}
\vspace{1cm}
ಸ್ಕ್ರಾಚ್ ಪ್ರಾರಂಭಿಸಿದಾಗ ಹೀಗೆ ಕಾಣಿಸುತ್ತದೆ. \par

ಪ್ರೋಗ್ರಾಂ ತಯಾರಿಯ ಸ್ಥಳದಲ್ಲಿ ಆಜ್ಞೆಗಳನ್ನು ಜೋಡಿಸಿ ಪ್ರೋಗ್ರಾಂ ತಯಾರಿಸಬೇಕು. ಪ್ರೋಗ್ರಾಂನಲ್ಲಿ ನೀಡಿದ ಆಜ್ಞೆಯ ಪ್ರಕಾರ ಸ್ಪ್ರೈಟ್ ಕೆಲಸ ಮಾಡುತ್ತದೆ. ಉದಾಹರಣೆಗೆ “ಹತ್ತು ಹೆಜ್ಜೆ ಮುಂದೆ ಹೋಗು” ಎಂದು ಆಜ್ಞೆ ಮಾಡಿದರೆ ಅದು ಹತ್ತು ಹೆಜ್ಜೆ ಮುಂದೆ ಹೋಗುತ್ತದೆ. ನಂತರ “90 ಡಿಗ್ರಿ ಬಲಕ್ಕೆ ತಿರುಗು” ಎಂದರೆ ಬಲಕ್ಕೆ ತಿರುಗುತ್ತದೆ. ಇನ್ನೂ ಹಲವಾರು ವಿಧದ ಆಜ್ಞೆಗಳಿವೆ. ಅವುಗಳನ್ನೆಲ್ಲ ಮುಂದಕ್ಕೆ ಒಂದೊಂದಾಗಿ ತಿಳಿಯೋಣ.
\vspace{1cm}


\chapter{ಮೌಸ್ ಬಳಕೆ}
%\begin{Large}\color{blue}{ಮೌಸ್ ಬಳಕೆ}
%\end{Large}
%}
\begin{enumerate}
\item{ಪ್ರೋಗ್ರಾಂ ತಯಾರಿಯ ಸ್ಥಳದಲ್ಲಿರುವ ಆಜ್ಞೆಯನ್ನು ಒತ್ತಿ.}
\item{ಪ್ರೋಗ್ರಾಂ ತಯಾರಿಯ ಸ್ಥಳದಲ್ಲಿರುವ ಆಜ್ಞೆಯನ್ನು ಅತ್ತಿತ್ತ ಜರುಗಿಸಿ.}
\item{ಪ್ರೋಗ್ರಾಂ ತಯಾರಿಯ ಸ್ಥಳದಲ್ಲಿರುವ ಆಜ್ಞೆಗೆ ಮತ್ತೊಂದು ಆಜ್ಞೆಯನ್ನು ಜೋಡಿಸಿ.}
\end{enumerate}
\begin{figure}[h]
\begin{Scratch}[1]
\beginbox{}
\scbox{ಲೇಖನಿಯುಕ್ತ}{pen}
\boucle{\cb[w]{10} ಮರುಕಳಿಸು}{4}{1}
\scbox{\cb[w]{10} ಹತ್ತು ಹೆಜ್ಜೆ ಮುಂದೆ ಹೋಗು}{motion}
\scbox{\cb[w]{1}  ಸೆಕೆಂಡುಗಳಷ್ಟು   ಕಾಯಬೇಕು}{control}
\scbox{\cb[w]{10} ಡಿಗ್ರಿಯಷ್ಟು ತಿರುಗು}{motion}
\turnbox{g}{180}{ಡಿಗ್ರಿಯಷ್ಟು ತಿರುಗು}
\boucle{\cb[w]{10} ಮರುಕಳಿಸು}{6}{1}
\scbox{\cb[w]{10} ಹತ್ತು ಹೆಜ್ಜೆ ಮುಂದೆ ಹೋಗು}{motion}
\scbox{\cb[w]{1}  ಸೆಕೆಂಡುಗಳಷ್ಟು   ಕಾಯಬೇಕು}{control}
\scbox{\cb[w]{10} ಡಿಗ್ರಿಯಷ್ಟು ತಿರುಗು}{motion}
\turnbox{g}{180}{ಡಿಗ್ರಿಯಷ್ಟು ತಿರುಗು}
\turnbox{1}{180}{ಡಿಗ್ರಿಯಷ್ಟು ತಿರುಗು}
\scbox{ಲೇಖನಿಮುಕ್ತ}{pen}
\end{Scratch}
\caption{ಉದಾಹರಣೆ ಪ್ರೋಗ್ರಾಂ 1}
\end{figure}

\Scratchy

\chapter{ಮೊದಲ ಪ್ರೋಗ್ರಾಂ}
\begin{figure}[h]
\begin{Scratch}[1]
\beginbox{}
\scbox{\cb[w]{10} ಹತ್ತು ಹೆಜ್ಜೆ ಮುಂದೆ ಹೋಗು}{motion}
\scbox{\cb[w]{1}  ಸೆಕೆಂಡುಗಳಷ್ಟು   ಕಾಯಬೇಕು}{control}
\scbox{\cb[w]{10} ಡಿಗ್ರಿಯಷ್ಟು ತಿರುಗು}{motion}
\end{Scratch}
\caption{ಉದಾಹರಣೆ ಪ್ರೋಗ್ರಾಂ 2}
\end{figure}


\section{ತಿರುಗುವುದು}

\section{ಮುಂದೆ ಹೋಗಿ ತಿರುಗುವುದು}

\section{ಕಾಯುವುದು}

\section{ಅಭ್ಯಾಸ }

\chapter{ಲೇಖನಿ – ಚಿತ್ರ ಬಿಡಿಸುವುದು}

\section{ಲೇಖನಿಯುಕ್ತ - ಮುಕ್ತ}
\begin{figure}[h]
\begin{Scratch}[1]
\beginbox{}
\scbox{\cb[w]{10} ಹತ್ತು ಹೆಜ್ಜೆ ಮುಂದೆ ಹೋಗು}{motion}
\scbox{\cb[w]{1}  ಸೆಕೆಂಡುಗಳಷ್ಟು   ಕಾಯಬೇಕು}{control}
\scbox{\cb[w]{10} ಡಿಗ್ರಿಯಷ್ಟು ತಿರುಗು}{motion}
\end{Scratch}
\caption{ಉದಾಹರಣೆ ಪ್ರೋಗ್ರಾಂ 3}
\end{figure}

\section{ಬಣ್ಣ ಬದಲಾಯಿಸುವುದು}

\section{ಗಾತ್ರ ಬದಲಾಯಿಸುವುದು}

\section{ಅಭ್ಯಾಸ }

\chapter{ಕಾಣುವುದು}

\section{ಹೆಲೊ ಎಂದು ಹೇಳು }

\section{ಶಬ್ದ }
\begin{figure}[h]
\begin{Scratch}[1]
\beginbox{}
\scbox{\cb[w]{10} ಹತ್ತು ಹೆಜ್ಜೆ ಮುಂದೆ ಹೋಗು}{motion}
\scbox{\cb[w]{1}  ಸೆಕೆಂಡುಗಳಷ್ಟು   ಕಾಯಬೇಕು}{control}
\scbox{\cb[w]{10} ಡಿಗ್ರಿಯಷ್ಟು ತಿರುಗು}{motion}
\end{Scratch}
\caption{ಉದಾಹರಣೆ ಪ್ರೋಗ್ರಾಂ 4}
\end{figure}

\section{ತೋರಿಸು, ಬಚ್ಚಿಡು}

\section{ಉಡುಪು ಬದಲಾಯಿಸು}

\section{ಅಭ್ಯಾಸ }

\chapter{ಮರುಕಳಿಸು (\textenglish{Loops})}
\begin{figure}[h]
\begin{Scratch}[1]
\beginbox{}
\scbox{\cb[w]{10} ಹತ್ತು ಹೆಜ್ಜೆ ಮುಂದೆ ಹೋಗು}{motion}
\scbox{\cb[w]{1}  ಸೆಕೆಂಡುಗಳಷ್ಟು   ಕಾಯಬೇಕು}{control}
\scbox{\cb[w]{10} ಡಿಗ್ರಿಯಷ್ಟು ತಿರುಗು}{motion}
\end{Scratch}
\caption{ಉದಾಹರಣೆ ಪ್ರೋಗ್ರಾಂ 5}
\end{figure}

\section{ಅಭ್ಯಾಸ }

\chapter{ಗ್ರಹಿಸುವುದು} 

\section{ಕೀಲಿಮಣೆ \textenglish{keyboard}}

\section{ ಮೌಸ್} 

\section{ಅಭ್ಯಾಸ }

\chapter{ಆಪರೇಟರ್ಗಳು}
\begin{figure}[h]
\begin{Scratch}[1]
\beginbox{}
\scbox{\cb[w]{10} ಹತ್ತು ಹೆಜ್ಜೆ ಮುಂದೆ ಹೋಗು}{motion}
\scbox{\cb[w]{1}  ಸೆಕೆಂಡುಗಳಷ್ಟು   ಕಾಯಬೇಕು}{control}
\scbox{\cb[w]{10} ಡಿಗ್ರಿಯಷ್ಟು ತಿರುಗು}{motion}
\end{Scratch}
\caption{ಉದಾಹರಣೆ ಪ್ರೋಗ್ರಾಂ 6}
\end{figure}

\section{ಅಂಕಗಣಿತದ ಆಯೋಜಕರು}
ಇದನ್ನು ಪರಿಶೀಲಿಸಲು ಮೇಲಿನ ಚಿತ್ರದಲ್ಲಿ ತೊರಿಸಿರುವಂತೆ ಆಜ್ಞೆಗಳನ್ನು ಎಳೆದು ತಂದು ಪ್ರೋಗ್ರಾಂ ತಯಾರಿಯ ಸ್ಥಳದಲ್ಲಿ ಇಡೋಣ. ಇದನ್ನು ಚಾಲತಿ ಮಾಡಿ ನೋಡಿ. “ಒಂದುವೇಳೆ” ಆಜ್ಞೆ ಸರಿ ಬರುವಂತೆ ಹಾಗು ತಪ್ಪಾಗುವಂತೆ ಮಾಡಿ ಪ್ರೋಗ್ರಾಂ ಚಾಲತಿಯಲ್ಲಿನ ವೆತ್ಯಾಸವನ್ನು ಮನವರಿಕೆ ಮಾಡಿಕೊಳ್ಳಿ. ಇಲ್ಲಿ, ಕೆಳಕಂಡ ಚಿತ್ರದಂತೆ, ಸ್ಪ್ರೈಟ್ ಕೆಲವು ಹೆಜ್ಜೆ ಮುಂದೆ ಹೋಗುವಂತೆ ಮಾಡಿದರೆ ಸ್ಪ್ರೈಟ್ ವರ್ತುಲಾಕಾರದಲ್ಲಿ ತಿರುಗುವುದನ್ನು ನೋಡಬಹುದು

\section{ಹೋಲಿಕೆ ಆಯೋಜಕರು}


\section{ತಾರ್ಕಿಕ ಆಯೋಜಕರು}

\section{ಅಭ್ಯಾಸ }

\chapter{ಒಂದುವೇಳೆ – ಇಲ್ಲದಿದ್ದರೆ(\textenglish{if-then-else})}
ನೆನಪಿರಲಿ, ಕಂಪ್ಯೂಟರ್ ಎಲ್ಲಾ ಆಜ್ಞೆಗಳನ್ನು ಅತಿ ವೇಗವಾಗಿ ಮಾಡಿ ಮುಗಿಸುತ್ತದೆ. ನಮಗೆ ಅದು ಕೆಲಸ ಮಾಡುವುದು ಹಂತ ಹಂತವಾಗಿ ಕಾಣಬೇಕಿದಲ್ಲಿ ಕಾಯುವ ಆಜ್ಞೆ ಮತ್ತು ಮರುಕಳಿಸುವ ಆಜ್ಞೆಗಳನ್ನು ಬಳಸಬೇಕು. ಕೆಳಕಂಡ ಚಿತ್ರಗಳಂತೆ, ಪ್ರೊಗ್ರಾಂ ಮಾಡಿ ನೋಡಿ. 
\section{ಒಂದುವೇಳೆ}

\section{ಒಂದುವೇಳೆ, ಇಲ್ಲದಿದ್ದರೆ}

\section{ವರೆಗೂ ಮರುಕಳಿಸು} 

\section{ಅಭ್ಯಾಸ }

\chapter{ಪಟ್ಟಿಗಳು}
\begin{figure}[h]
\begin{Scratch}[1]
\beginbox{}
\scbox{\cb[w]{10} ಹತ್ತು ಹೆಜ್ಜೆ ಮುಂದೆ ಹೋಗು}{motion}
\scbox{\cb[w]{1}  ಸೆಕೆಂಡುಗಳಷ್ಟು   ಕಾಯಬೇಕು}{control}
\scbox{\cb[w]{10} ಡಿಗ್ರಿಯಷ್ಟು ತಿರುಗು}{motion}
\end{Scratch}
\caption{ಉದಾಹರಣೆ ಪ್ರೋಗ್ರಾಂ 7}
\end{figure}

\section{ಪಟ್ಟಿಗಳನ್ನು ಬಳಸುವುದು}

\section{ಪಟ್ಟಿಗಳನ್ನು ಬದಲಾಯಿಸುವುದು} 

\section{ಅಭ್ಯಾಸ }

\chapter{ಭೌತಿಕ ಗಣಕ}

\section{ಎಲ್.ಇ.ಡಿ ಉರಿಸುವುದು}
\begin{figure}[h]
\LED{red}\LED{green}\LED{yellow}\LED{gray}
\caption{ಉದಾಹರಣೆ ಪ್ರೋಗ್ರಾಂ ಎಲ್.ಇ.ಡಿ}
\end{figure}

\section{ಗ್ರಹಿಸಿ ಪ್ರತಿಕ್ರಿಯಿಸುವುದು} 
\begin{figure}[h]
\begin{Scratch}[1]
\beginbox{}
\scbox{\cb[w]{10} ಹತ್ತು ಹೆಜ್ಜೆ ಮುಂದೆ ಹೋಗು}{motion}
\scbox{\cb[w]{1}  ಸೆಕೆಂಡುಗಳಷ್ಟು   ಕಾಯಬೇಕು}{control}
\scbox{\cb[w]{10} ಡಿಗ್ರಿಯಷ್ಟು ತಿರುಗು}{motion}
\end{Scratch}
\caption{ಉದಾಹರಣೆ ಪ್ರೋಗ್ರಾಂ 8}
\end{figure}

\section{ಅಭ್ಯಾಸ }

\end{document}